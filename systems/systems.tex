\documentclass{beamer}

\newcommand{\RR}{\mathbb{R}}
\newcommand{\cS}{\mathcal{S}}

\begin{document}
The systems 
\[
\begin{array}{llrrrrrrrrrl}
&&x_1 & - & 2x_2 & - & 7x_3 & = & -1 &&& E_1\\
&&-x_1 & + & 3x_2 & + & 6x_3 & = & 0 &&& E_2\\
\text{and}\\
&&x_1 & - & 2x_2 & - & 7x_3 & = & -1&&&E_1\\
 &&&  & x_2 & - & x_3 & = &-1&&&E_2+E_1 
\end{array}
\]
have the same solutions:

\bigskip
Let $\vec{x}=(x_1,x_2,x_3)$.
\[
\text{$\vec{x}$ a solution of $E_1$ and $E_2$} \Longrightarrow
\text{$\vec{x}$ a solution of ($E_1$ and) $E_1+E_2$}.
\]
Since $E_1 = (E_1+E_2)-E_2$,
\[
\text{$\vec{x}$ a solution of $E_1+E_2$ and $E_2$} \Longrightarrow
\text{$\vec{x}$ a solution of $E_1$ (and $E_1+E_2$)}
\]

\frame{
Similarly, the systems 
\[
\begin{array}{llrrrrrrrrrl}
&&x_1 & - & 2x_2 & - & 7x_3 & = & -1&&&E_1\\
 &&&  & x_2 & - & x_3 & = &-1&&&E_2+E_1\\
  \text{and}\\
&&x_1 &  &  & - & 9x_3 & = & -3&&&E_1+2(E_2+E_1)\\
 &&&  & x_2 & - & x_3 & = &-1&&&E_2+E_1 
\end{array}
\]
have the same solutions.

\bigskip
Psychological step: Stop thinking of $x_3$ as a variable. Rather, think of it as a \textbf{parameter}. 

\bigskip
Distinguish variables and the parameter notationally: Set $t=x_3$. 
\[
\begin{array}{llrrrrrrrl}
&&x_1 &  &  &  = & 9t-3&&&E_1+2(E_2+E_1)\\
 &&&  & x_2 &  = &t-1&&&E_2+E_1 
\end{array}
\]
Conclusion: $(x_1,x_2,x_3)=(9-3t,t-1,t)$ is a solution \textbf{for all $t$}.
}

\frame{
	In particular, the system has \textbf{infinitely many solutions}.

	\bigskip
	Analyze the above argument: All solutions have the form
	\[
	(x_1,x_2,x_3)=(9-3t,t-1,t),\tag{$*$}
	\]
	for some $t$.

	\bigskip
	$(*)$ is called the \textbf{general solution} of the system.
}
\end{document}